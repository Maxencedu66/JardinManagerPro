\documentclass[french, 12pt]{article}

\usepackage[utf8]{inputenc}
\usepackage[T1]{fontenc}
\usepackage[french]{babel}
\usepackage{array,multirow,makecell}
\setcellgapes{1pt}
\makegapedcells
\newcolumntype{R}[1]{>{\raggedleft\arraybackslash }b{#1}}
\newcolumntype{L}[1]{>{\raggedright\arraybackslash }b{#1}}
\newcolumntype{C}[1]{>{\centering\arraybackslash }b{#1}}

\usepackage{graphicx} % Cette librairie permet d'afficher des images


\input{./commandes.tex}%on importe le fichier de commandes
\graphicspath{{../Images/}}%le chemin qui mène au dossier d'images

\author{Thomas Jeanjacquot}
\date{13 Octobre 2022}
\title{Compte-rendu }


\begin{document}




\begin{center}
\large{ \textbf{Compte-rendu n°01}


\underline{\textbf{Réunion du 13 Octobre 2022}} \\
\bigskip}

\begin{tabular}{|L{8cm}|L{8cm}|}
\hline Motif/type de la réunion: Réunion de lancement & Lieu: Hall de Télécom \\
\hline Heure de début: 16h & Heure de fin: 16h45 \\
\hline

\end{tabular}

\end{center}


\paragraph{Objet:}
Lancement du projet

\bigskip
\noindent
\begin{tabular}{|L{8cm} L{8cm}|}
\hline Présents: & \\
\hline
BOUCHADEL Maxence & GONCALVES Florian \\
JEANJACQUOT Thomas & VATRY Étienne\\
\hline

\end{tabular}

\section*{Ordre du jour:}
\begin{itemize}
    \item Planifier une première réunion technique
    \item Se mettre d'accord sur le sujet du projet
\end{itemize}


\section*{Échanges:}
Le groupe a d'abord lu ensemble le sujet du projet disponible sur arche ainsi que les différentes notes prises par chacun afin de bien s'accorder sur ce qu'il fallait réaliser.

Nous nous sommes ensuite fixé un objectif commun: effectuer chacun des recherches sur les applications et solutions déjà existantes afin de pouvoir les mettre en commun lors de la prochaine réunion et ainsi d'être capable de réaliser un état de l'art.

Nous avons de plus réfléchi aux différents moyens de communication que nous pouvons mettre en place afin de fluidifier nos échanges. Nous avons alors créer un groupe Discord afin de pouvoir facilement communiquer des fichiers, réflexions et liens rapidement à tout le groupe.

Pour finir nous avons mis en commun nos emplois du temps afin de fixer les prochaines réunions du groupe.


\subsection*{To Do List:}
\noindent
\begin{tabular}{|L{12cm}|L{4cm}|}
\hline Description & Délai \\
\hline Faire des recherches sur le numérique dans les circuits courts et dans les jardins partagés (afin de préparer l'état de l'art) & Mardi 18/10 \\
\hline Réfléchir à des idées d'application & Mardi 18/10\\
\hline 

\end{tabular}

\paragraph{Prochaine Réunion:}
Mardi 18 Octobre de 12H45 à 14H



 


 
 

\end{document}
